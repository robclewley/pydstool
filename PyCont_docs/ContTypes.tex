% ***********
% NOTES
% ***********
% 
% BIO220
% 	* For Life Tables, see Sebastion's Calc book (End of Chapter 7)
%
\documentclass{article}[12pt]

%%
%% MATH DEFINITIONS   
%%

\def\({\left(}
\def\){\right)}

\def\inv#1{{\frac{1}{#1}}}
\def\ddt#1{\der{#1}{t}}
\def\ddx#1{\der{#1}{x}}
\def\ddz#1{\der{#1}{z}}
\def\der#1#2{{\frac{d#1}{d#2}}}
\def\derder#1#2{{d^2\frac{#1}{d{#2}^2}}}
\def\norm#1{\labs #1 \rabs}
\def\pder#1#2{{\frac{\partial#1}{\partial#2}}}
\def\pderder#1#2{{\frac{\partial^2#1}{\partial{#2}^2}}}
\def\pdt#1{\pder{#1}{t}}
\def\pdx#1{\pder{#1}{x}}
\def\pdxx#1{\pderder{#1}{x}}
\def\pdz#1{\pder{#1}{z}}
\def\pdzz#1{\pderder{#1}{z}}
\def\ssize#1{{\scriptsize #1}}

\def\bmat{\left[ \begin{array}}
\def\bna{\begin{array}}
\def\bneas{\begin{eqnarray*}}
\def\bnea{\begin{eqnarray}}
\def\bnes{\begin{displaymath}}
\def\bne{\begin{equation}}

\def\emat{\end{array} \right]}
\def\ena{\end{array}}
\def\eneas{\end{eqnarray*}}
\def\enea{\end{eqnarray}}
\def\enes{\end{displaymath}}
\def\ene{\end{equation}} 

\def\half{\frac{1}{2}} 

\def\revrxn#1#2{
\begin{array}{c} 
{#1}\\
\rightleftharpoons \\
{#2}
\end{array}
}

\def\Bold#1{\vspace{0.1in} \noindent {\bf \boldmath #1}}
\def\Eq#1{Eq.~\ref{#1}}
\def\Eqs#1#2{Eqs.~\ref{#1}--\ref{#2}}
\def\Fig#1{Fig.~\ref{#1}}
\def\avg#1{\langle{#1}\rangle}
\def\Parens#1{\left({#1}\right)}



%%
%% CALCIUM DEFINITIONS   
%%

% My Definitions

\def\FcepsilonR1{Fc$\epsilon$R1}

\def\Ps{$\ps$}
\def\ps{{\rm s}^{-1}}

\def\Pms{$\pms$}
\def\pms{{\rm ms}^{-1}}

\def\UM{$\uM$}
\def\uM{\mu{\rm M}}

\def\Um{$\um$}
\def\um{\mu{\rm m}}

\def\Us{$\us$}
\def\us{\mu{\rm s}}

\def\Ums{$\ums$}
\def\ums{\mu{\rm m}^2}

\def\Nm{$\nm$}
\def\nm{\mu{\rm m}}

\def\Pumps{$\pumps$}
\def\pumps{\mu{\rm M}^{-1} {\rm s}^{-1}}

\def\Pumpms{$\pumpms$}
\def\pumpms{\mu{\rm M}^{-1} {\rm ms}^{-1}}

\def\UMps{$\uMps$}
\def\uMps{\mu{\rm M}/{\rm s}}

\def\Umpss{$\umpss$}
\def\umpss{\mu{\rm m}/{\rm s}^2}

\def\Umsps{$\umsps$}
\def\umsps{\mu{\rm m}^2/{\rm s}}

\def\Umspms{$\umspms$}
\def\umspms{\mu{\rm m}^2/{\rm ms}}

\def\Ryr{RyR}
\def\ip{{[{\rm IP}_3]}}

\def\Ip{${\rm IP}_3$}
\def\Ipr{${\rm IP}_3{\rm R}$}

\def\cad{[{\rm Ca}^{\rm 2+}]_{\rm d}}
\def\Cad{$\cad$}

\def\cabulk{[{\rm Ca}^{\rm 2+}]_{\rm bulk}}
\def\Cabulk{$\cabulk$}

\def\cainf{[{\rm Ca}^{\rm 2+}]_{\infty}}
\def\Cainf{$\cainf$}

\def\cai{[{\rm Ca}^{\rm 2+}]_{\rm i}}
\def\Cai{$\cai$}

\def\caext{[{\rm Ca}^{\rm 2+}]_{\rm ext}}
\def\Caext{$\caext$}

\def\cae{[{\rm Ca}^{\rm 2+}]_{\rm ER}}
\def\Cae{$\cae$}

\def\cat{[{\rm Ca}^{\rm 2+}]_{\rm T}}
\def\Cat{$\cat$}

\def\ve{{\varepsilon}}
\def\Bj{${\rm B}_j$}
\def\B{${\rm B}$}
\def\Bt{$\bt$}
\def\Ca{${\rm Ca}^{\rm 2+}$}
\def\K{${K}^{+}$}
\def\Cabj{${\rm CaB}_j$}
\def\Cab{${\rm CaB}$}
\def\Kj{$\kj$}
\def\Kjm{$\kjm$}
\def\Kjp{$\kjp$}
\def\Kmm{$\kmm$}
\def\Kkm{$\kkm$}
\def\Kmp{$\kmp$}
\def\Km{$\km$}
\def\Kp{$\kp$}
\def\bjt{\bj_T}
\def\b{[{\rm B}]}
\def\bj{[{\rm B}_j]}
\def\bmt{\bm_T}
\def\bt{\b_{\rm T}}
\def\bm{[B_m]}
\def\cabj{[{\rm CaB}_j]}
\def\cab{[{\rm CaB}]}
\def\ca{[{\rm Ca}^{2+}]}
\def\dc{D_c}
\def\dj{D_j}
\def\dm{D_m}
\def\db{D_b}
\def\kj{K_j}
\def\kjm{k^-_j}
\def\kjp{k^+_j}
\def\kmm{k^-_m}
\def\kmp{k^+_m}
\def\kkm{K_m}
\def\km{k^-}
\def\kp{k^+}

\def\wm{w^-}
\def\wp{w^+}
\def\minf{m_\infty}

% partitioning
\def\CO{\mathcal{C/O}}
\def\CRO{\mathcal{C/R/O}}

% other ions

\def\mg{[{\rm Mg}^{2+}]}
\def\Mg{${\rm Mg}^{\rm 2+}$}





\textwidth 6in
\oddsidemargin 0.25in
\evensidemargin 0.25in
\textheight 8in

\usepackage{fancybox}
\usepackage{epsfig}
\usepackage{graphicx}% Include figure files
\usepackage{color}
\usepackage{dcolumn}% Align table columns on decimal point
\usepackage{bm}% bold math
%\usepackage{amsthm}
\usepackage{amsmath}
\usepackage{amssymb}
\usepackage{amsxtra}
\usepackage{cancel}
\usepackage{multirow}
\usepackage{undertilde}
\usepackage{calc}
\usepackage{stackrel}
\usepackage{hyperref}
\hypersetup{
    colorlinks,%
    citecolor=magenta,%
    filecolor=magenta,%
    linkcolor=magenta,%
    urlcolor=blue
}

\newtheorem{theorem}{Theorem}[section]
\newtheorem{definition}[theorem]{Definition}
\newtheorem{cor}[theorem]{Corollary}
\newtheorem{lem}[theorem]{Lemma}
\newtheorem{prop}[theorem]{Proposition}
\newtheorem{rem}[theorem]{Remark}
\newtheorem{conj}[theorem]{Conjecture}
\newtheorem{ques}[theorem]{Question}
\newtheorem{alg}[figure]{Algorithm}

\renewcommand{\setminus}{-}
\newcommand{\swtw}[4]{\ensuremath{\left(\begin{array}{c} #1 \ #3 \\ #2 \ #4 \end{array}\right)}}

\newcommand{\swth}[6]{\ensuremath{\left(\begin{array}{c} #1 \ #3 \ #5 \\ #2 \ #4 \ #6 \end{array}\right)}}

% Test functions
%\renewcommand{\theequation}{LP-T\arabic{equation}}

\def\bF{\bm{F}}
\def\bV{\bm{V}}
\def\by{\bm{y}}
\def\bty{\bm{\tilde{y}}}
\def\ba{\bm{a}}
\def\bta{\bm{\tilde{a}}}
\def\bzero{\bm{0}}
\def\coloneqq{\mathrel{\mathop:}=}

\newenvironment{whitebox}[1][0pt]{
\def\FrameCommand{\psshadowbox[shadowsize=0.3em, framesep=1.0em, fillstyle=solid, fillcolor=white]}
\MakeFramed{\advance\hsize-\width \advance\hsize-#1 \FrameRestore}}
{\endMakeFramed}

\setcounter{secnumdepth}{5}
\setcounter{tocdepth}{5}
 
\newenvironment{Shadowed}{\begin{center}\begin{shadow}\begin{minipage}{.8\textwidth}}{\end{minipage}\end{shadow}\end{center}}

%\makeatletter
%\newcommand\Paragraph[1]{%
%  \addcontentsline{prg}{Paragraph}{#1}%
%  \paragraph{#1}}
%\newcommand\l@Paragraph[2]{#1,~\textit{#2}}
%\newcommand\listParagraphname{\large{-- List of Bifurcation Points --}}
%\newcommand\listofParagraphs{%
%  \section*{\listParagraphname}\@starttoc{prg}}
%\makeatother
%\newcommand\AddSep{\addtocontents{prg}{\newline}}

\title{PyCont: Continuation Types}

\pagestyle{empty}

\begin{document}

\date{}
\maketitle

\tableofcontents

\newpage
\section{Introduction and notation}

Consider the following differential equation:
\[
\frac{d\by}{dt} = \bF(\by,\ba),
\]
where $\by = (y_{1},y_{2},\ldots,y_{n})$ are phase variables, $\ba = (a_{1},a_{2},\ldots,a_{m})$ are parameters, and $\bF = (F_{1}(\bm{y},\bm{a}),\ldots,F_{n}(\bm{y},\bm{a}))$ are $n$ functions.  The jacobian of $\bF$ will be denoted by $\bF_{\by}$.

\begin{center}
\shadowbox{
\begin{minipage}{0.8\textwidth}
PyCont: The function $\bF$ is stored in the variable \texttt{self.sysfunc}.
\end{minipage}
}
\end{center}

\section{Bordered Matrix Methods (\texttt{class BorderMethod(TestFunc)})}
\label{sec:bord}

Suppose we have a test function that signals a bifurcation point when $\det(A) = 0$, where $A$ is an $n\times n$ matrix.  We can consider the bordered extension $M$ of $A$ given by
\[
M = \left(\begin{array}{cc}
A & b \\
c^{T} & d
\end{array}\right),
\]
where $b,c\in \mathbb{R}^{n}$ and $d\in \mathbb{R}$.  If we suppose that $M$ is nonsingular and we solve the system
\[
M\left(\begin{array}{c} r \\ s \end{array}\right) = \left(\begin{array}{c} 0_{n} \\ 1 \end{array}\right),
\]
where $r\in\mathbb{R}^{n}$ and $s\in \mathbb{R}$, then by Cramer's rule we have
\[
s = \frac{\det(A)}{\det(M)}.
\]
Thus, $\det(A) = 0$ if and only if $s(A) = 0$.

In the general case, we have
\begin{equation}
\left(\begin{array}{cc}
A & B \\
C^{T} & D
\end{array}\right)
\left(\begin{array}{c}
V \\
G
\end{array}\right) = 
\left(\begin{array}{c}
0 \\
1
\end{array}\right)
\label{eq:bord1}
\end{equation}
where
\begin{eqnarray*}
r & = & \mbox{corank} \\
A & = & n\times m \\
s & = & \max(n,m) \\
p & = & s-m+r \\
q & = & s-n+r \\
B & = & n\times p \\
C & = & m\times q \\
D & = & 0_{q\times p} \\
V & = & m\times q \\
G & = & p\times q \\
0 & = & n\times q \\
1 & = & q\times q.
\end{eqnarray*}
The matrix $G$ can also be seen as a function of $A$ as
\[
G_{\mathrm{bor}}:\mathbb{R}^{nm} \rightarrow \mathbb{R}^{pq} \qquad \mbox{such that} \qquad G_{\mathrm{bor}}(A) = G.
\]
It can also be written as
\begin{equation}
\left(\begin{array}{cc} W^{T} & G \end{array}\right)
\left(\begin{array}{cc}
A & B \\
C^{T} & D
\end{array}\right) =
\left(\begin{array}{cc} 0 & 1 \end{array}\right),
\label{eq:bord2}
\end{equation}
where now
\begin{eqnarray*}
W & = & n\times p \\
0 & = & p\times m \\
1 & = & p\times p.
\end{eqnarray*}
This is important for calculating derivatives, since we have
\[
G_{z} = -W^T A_{z} V.
\]

\subsection{Initialization (\texttt{BorderMethod.setdata})}

In the bordered matrix methods, we need to initialize the $B$ and $C$ matrices so that the matrix $M$ is nonsingular.  Suppose we have the singular value decomposition of $A$ as $A = U\Sigma Z^{T}$, where $U$ is $n\times t$, $\Sigma$ is $t\times t$, $Z$ is $m\times t$ and $t = \min(n,m)$.  We initialize $B$ and $C$ as follows:
\begin{eqnarray*}
B & = & \left(\begin{array}{ccc} U_{t-p+1} & \cdots & U_{t} \end{array}\right) \\
C & = & \left(\begin{array}{ccc} Z_{t-q+1} & \cdots & Z_{t} \end{array}\right)
\end{eqnarray*}
Note that $p, q \leq t$.

\subsection{Function evaluation (\texttt{BorderMethod.func})}

By using the LU factorization of $M$, we can solve \eqref{eq:bord1} and \eqref{eq:bord2} for $V$, $W$ and $G$.  We then update the matrices $B$ and $C$ as follows:
\begin{eqnarray*}
B & = & ||A||_{1}\frac{W}{||W||_{1}}, \\
C & = & ||A||_{\infty}\frac{V}{||V||_{1}}.
\end{eqnarray*}

\section{Codimension 1}

\subsection{Continuous Dynamical Systems}

\subsubsection{Equilibrium Curves (EP-C) (\texttt{class EquilibriumCurve(Continuation)})}

%\listofParagraphs
%\bigskip

\paragraph{Mathematical definition}

In this case, we are concerned with curves of {\it equilibrium points} (EP) as a function of a free parameter $a_{1}$, defined by
\[
\bF(\bty,\bta) = \bzero,
\]
where $\bF:\mathbb{R}^{n+1} \rightarrow \mathbb{R}^{n}$, $\bty = (y_{1},\ldots,y_{n},a_{1})$ and $\bta = (a_{2},\ldots,a_{m})$.

\begin{center}
\shadowbox{
\begin{minipage}{.8\textwidth}
The phase variables $(y_{1},\ldots,y_{n})$ are stored in \texttt{self.coords}, while the free parameter $a_{1}$ is stored in \texttt{self.params}.
\end{minipage}
}
\end{center}

The jacobian is given by $\bF_{\bty} : \mathbb{R}^{n+1} \rightarrow \mathbb{R}^{n}$, where
\[
\bm{F}_{\bty}(\bty,\bta) = \left(\bF_{\by} \ \bF_{a_{1}}\right).
\]

\paragraph{Detection of bifurcation points}

We have the following bifurcation points on an equilibrium curve:
\begin{itemize}
\item Branch Bifurcation Point (BP) (\texttt{class BranchPoint(BifPoint)})
\item Fold Bifurcation Point (LP) (\texttt{class FoldPoint(BifPoint)})
\item Hopf Bifurcation Point (H) (\texttt{class HopfPoint(BifPoint)})
\end{itemize}
To detect these bifurcation points, we use the following test functions:
\begin{eqnarray}
\phi_{1}(\bty) = & \det\left(\begin{array}{c} \bF_{\bty} \\ \bV^{T} \end{array}\right) \qquad & \mbox{(\texttt{Branch\_Det})} \label{eq:BP-T1} \\
\phi_{2}(\bty) = & V_{n+1} \qquad & \mbox{(\texttt{Fold\_Tan})} \label{eq:LP-T1} \\
\phi_{3}(\bty) = & G_{\mathrm{bor}}(2\bF_{\by} \odot I_{n}) \qquad & \mbox{(\texttt{Hopf\_Bor})} \label{eq:H-T1}
\end{eqnarray}
\begin{center}
    \begin{tabular}{ | c | c | c | c |}
    \hline
     & $\phi_{1}$ & $\phi_{2}$ & $\phi_{3}$ \\ \hline
    BP & 0 & - & - \\ \hline
    LP & 1 & 0 & - \\ \hline
    H  & - & - & 0 \\ \hline
    \end{tabular}
\end{center}
In the table above, a zero and a one corresponds to the test functions being zero or nonzero, respectively.  Alternate test functions include:
\begin{eqnarray*}
\phi(\bty) = & \det(\bF_{\by}) \qquad & \mbox{(\texttt{Fold\_Det})} \label{eq:LP-T2} \\
\phi(\bty) = & G_{\mathrm{bor}}(\bF_{\by}) \qquad & \mbox{(\texttt{Fold\_Bor})} \label{eq:LP-T3} \\
\phi(\bty) = & \det(2\bF_{\by} \odot I_{n}) \qquad & \mbox{(\texttt{Hopf\_Det})} \label{eq:H-T2} \\
& \mbox{\textit{NOT USED}}  \qquad & \mbox{(\texttt{Hopf\_Eig})} \label{eq:H-T3}
\end{eqnarray*}

\paragraph{Location of bifurcation points (general)}

%\begin{figure}[h!]
\begin{center}
\shadowbox{
\ttfamily\upshape

\begin{tabular}{l} \hline\hline
{\bf Algorithm:} Locate zeros of test functions \\ \hline
{\bf Input:} Two points on the curve given by $(x_{1},v_{1})$ and $(x_{2},v_{2})$ such that \\
\quad $\phi_{1}(x_{1},v_{1}) < 0$ and $\phi_{1}(x_{2},v_{2}) > 0$ \\ 
{\bf Output:} Found point $(x,v)$ \\ \\
$\Phi_{1} \coloneqq \phi_{1}(x_{1},v_{1})$ \\
$\Phi_{2} \coloneqq \phi_{1}(x_{2},v_{2})$ \\
{\bf for} $i \coloneqq$ 1 to MaxTestIters \\
\qquad $r \coloneqq \left|\frac{\Phi_{1}}{\Phi_{1}-\Phi_{2}}\right|$ \\
\qquad {\bf if} $r \geq 1$ \\
\qquad\qquad $r = 0.5$ \\
\qquad $x \coloneqq x_{1} + r(x_{2}-x_{1})$ \\
\qquad $v \coloneqq v_{1} + r(v_{2}-v_{1})$ \\
\qquad $(x,v) \coloneqq$ Corrector$\bigl((x,v)\bigr)$ \\
\qquad $\Phi \coloneqq \phi_{1}(x,v)$ \\
\qquad {\bf if} $|T| <$ TestTol {\bf and} $\min\left(|x-x_{1}|,|x-x_{2}|\right) <$ VarTol \\
\qquad\qquad {\bf break} \\
\qquad {\bf else} \\
\qquad\qquad {\bf if} $\mathrm{sign}(\Phi) == \mathrm{sign}(\Phi_{2})$ \\
\qquad\qquad\qquad $(x_{2},v_{2},\Phi_{2}) \coloneqq (x,v,\Phi)$ \\
\qquad\qquad {\bf else} \\
\qquad\qquad\qquad $(x_{1},v_{1},\Phi_{1}) \coloneqq (x,v,\Phi)$ \\
{\bf return} $(x,v)$ \\
\hline\hline
\end{tabular}
\rmfamily
}
\end{center}

\paragraph{Location of branch points (\texttt{class BranchPoint(BifPoint).locate})}
\label{sec:locateBP}

As mentioned in MATCONT, the region of attraction near a BP point has the shape of a cone, which we cannot guarentee to stay within.  We thus define temporary variables $\beta\in\mathbb{R}$ and $p\in\mathbb{R}^{n}$ and implement Newton's method in the space $(\bty,\beta,p)\in\mathbb{R}^{2(n+1)}$ with the extended system given by:
\begin{equation}
\left\{\begin{array}{rcc}
\bF(\bty,\bta) + \beta p & = & 0 \\
\left[\bF_{\by}(\bty,\bta) \ \bF_{a_{1}}(\bty,\bta)\right]^{T}p & = & 0 \\
p^{T}p - 1 & = & 0
\end{array}\right.
\end{equation}
We start with $\beta=0$ and $p$ the left eigenvector of $\bF_{\by}$ associated with the smallest eigenvalue.

\subparagraph{Computation of branch direction}

NOTE: Doesn't currently work!!!!!  See \cite{Beyn_numericalcontinuation} for mathematical discussion, notes in NoteTakerHD and PyCont\_Brusselator.py for example code.

Setting $\psi$ to the $p$ found upon convergence in the above Newton's method, we first set $V_{1}$ to the real part of the eigenvector associated with the smallest (i.e. zero) eigenvalue of the matrix associated with the test function \eqref{eq:BP-T1}
\[
\left(\begin{array}{c} \bF_{\bty} \\ \bV^{T} \end{array}\right).
\]
Then, given the Hessian $H$ of $\bF(\bty)$, we compute the following scalars:
\begin{eqnarray*}
c_{11} & = & \psi^{T}H[V,V] \\
c_{12} & = & \psi^{T}H[V,V_{1}] \\
c_{22} & = & \psi^{T}H[V_{1},V_{1}] \\
\beta & = & 1 \\
\alpha & = & -\frac{c_{22}}{2c_{12}} 
\end{eqnarray*}
We then compute the direction of the new branch as
\[
V_{\mathrm{new}} = \alpha V + \beta V_{1}.
\]
{\bf Note:} $c_{11}$ is not used in the computation but is included for completeness.  Also, $\beta = 1$ and so can be omitted.  (I need to find reference for this!)

\subsection{Discrete Dynamical Systems}

\section{Codimension 2}

\subsection{Continuous Dynamical Systems}

\subsubsection{Fold Curves (LP-C) (\texttt{class FoldCurve(Continuation)})}

In this case, we are concerned with curves of {\it fold bifurcation points} (LP) as a function of two free-parameters $(a_{1},a_{2})$, defined by the augmented system
\begin{equation}
\bm{C}(\bty,\bta) = \left\{\begin{array}{c} \bF(\bty,\bta) \\ G_{\mathrm{bor}}(\bF_{\by}) \end{array}\right. = \bzero,
\label{eq:LP-C}
\end{equation}
such that $\bm{C}:\mathbb{R}^{n+2} \rightarrow \mathbb{R}^{n+1}$, $\bty = (y_{1},\ldots,y_{n},a_{1},a_{2})$ and $\bta = (a_{3},\ldots,a_{m})$.  We have the following bifurcation points on a fold curve:
\begin{itemize}
\item Bogdanov-Takens (BT) (\texttt{class BTPoint(BifPoint)})
\item Zero-Hopf point (ZH) (\texttt{class ZHPoint(BifPoint)})
\item Cusp point (CP) (\texttt{class CPPoint(BifPoint)})
\item Branch point (BP) (\texttt{class BranchPoint(BifPoint)})
\end{itemize}
For the bordered method \texttt{Fold\_Bor} in \eqref{eq:LP-C}, we have $p=q=1$, and thus the vectors $v=V$ and $w=W$ in equations \eqref{eq:bord1} and \eqref{eq:bord2} are both $n\times 1$.  They are updated continuously throughout the continuation, and are used in the test functions for these bifurcation points as follows:
\begin{eqnarray}
\phi_{1}(\bty) = & w^{T}v \qquad & \mbox{(\texttt{BT\_Fold})} \label{eq:BT-T1} \\
\phi_{2}(\bty) = & G_{\mathrm{bor}}(2\bF_{\by} \odot I_{n}) \qquad & \mbox{(\texttt{Hopf\_Bor}, \eqref{eq:H-T1})} \nonumber \\
\phi_{3}(\bty) = & w^{T}\bF_{\by\by}[v,v] \qquad & \mbox{(\texttt{CP\_Fold})} \label{eq:CP-T1} \\
\phi_{4}(\bty) = & w^{T}\left[\bF_{a_{1}} \ \bF_{a_{2}}\right] \qquad & \mbox{(\texttt{BP\_Fold})} \label{eq:BPF-T1}
\end{eqnarray}
\begin{center}
    \begin{tabular}{ | c | c | c | c | c | c | }
    \hline
     & $\phi_{1}$ & $\phi_{2}$ & $\phi_{3}$ & $\phi_{4,1}$ & $\phi_{4,2}$ \\ \hline
    BT & 0 & 0 & - & - & - \\ \hline
    ZH & 1 & 0 & - & - & - \\ \hline
    CP & - & - & 0 & - & - \\ \hline
    BP & - & - & - & 0 & - \\ \hline
    BP & - & - & - & - & 0 \\ \hline 
    \end{tabular}
\end{center}

For an example of branch points on a fold curve, see PyCont\_BranchFold.py.

\bibliography{ContTypes}
\bibliographystyle{plain}

\end{document}